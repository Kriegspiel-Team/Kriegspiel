%\section{Analyse de l'existant}  
\chapter{Analyse de l'existant}
\markboth{\MakeUppercase{Analyse de l'existant}}{} 

	\paragraph{}
	Le jeu de la guerre appelé aussi "Kriegspiel" est parfois considéré comme une évolution du jeu d'échecs.
	Le modèle qui nous intéresse et sur lequel nous nous baserons dans le cadre de ce projet est celui de Guy Debord qui l'a breveté en 1965.
	Une application permettant de jouer au jeu a été développée par Radical Software Group (RSG), un collectif de développeurs et d'artistes 
	créé en 2000 dont le but est de travailler sur des programmes expérimentaux.
	
	\paragraph{}
	Cette version du jeu a été développée en Java en utilisant les librairies suivantes :
	
	\begin{itemize}
		\item Java Monkey Engine pour le moteur de jeu
		\item Project Darkstar pour la partie réseau
		\item La modélisation des objets graphiques a été faite via Blender
	\end{itemize}

	\paragraph{}
	L'interface du jeu a été travaillée de sorte à montrer aux joueurs les déplacements possibles pour chaque pièce. De plus les axes de 
	communication sont affichés avec des pointillés pour faciliter la vision du jeu.
	
	\paragraph{}
	L'application propose un mode "practice" permettant de jouer seul afin de se familiariser avec les règles du jeu ainsi que l'interface.
	Un second mode permet aux joueurs de s'affronter en ligne afin d'expérimenter de nouvelles stratégies.
	
	\paragraph{}
	Les développeurs du site RSG ont écarté l'idée d'y implémenter un joueur automatique puisque le but de leur projet était de 
	pouvoir y apprendre la stratégie face a un adversaire réel.
	
	\paragraph{}
	Malgré les documents décrivant des stratégies de jeu, il n'existe donc pas actuellement de joueur automatique pour ce jeu.
	
	\clearpage