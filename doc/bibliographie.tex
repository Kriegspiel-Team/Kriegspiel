\section{Elements bibliographiques}

	\begin{itemize}

		\item Le livre~\cite{ref1} présente une partie commentée du jeu de la guerre accompagnée de schémas illustrés. 
		De plus on peut trouver les règles officielles proposées par Guy Debord à partir de la page 131.
		\\[0.7\baselineskip]

		\item Le livre Artificial Intelligence for Games~\cite{ref2} aborde dans son 5\up{ème} chapitre la prise de décision d'un point de vue technique. 
		On y trouve le principe de fonctionnement et un exemple d'implémentation de diverses structures de données (arbre de décision, machine à état, 
		arbre de comportement, moteur de règles).
		\\[0.7\baselineskip]

		\item Ce site~\cite{ref3} regroupe l'ensemble des règles du jeu de la guerre et propose une version jouable. Cette version jouable permettra 
		de bien comprendre les règles du jeu et de vérifier que notre jeu a bien le même comportement.
		\\[0.7\baselineskip]

		\item Cet ouvrage~\cite{ref4} concernant l'intelligence artificielle propose une grande variété d'idées sur les programmes heuristiques. 
		Il explique comment gérer la prise de décision acceptable mais non optimale, dans les cas où la prise de décision optimale nécessite trop de ressources.
		\\[0.7\baselineskip]

		\item Cette page~\cite{ref5} propose des explications d'algorithmes utilisables dans le cas de jeu se jouant à deux, tour par tour et 
		pour lequel chaque joueur connait la position de son adversaire.
		\\[0.7\baselineskip]

		\item Cette page~\cite{ref6} propose un tutoriel très complet sur l'utilisation de Jboss drools, outil qui nous permettra de concevoir notre moteur de règles.
		\\[0.7\baselineskip]

		\item Ce tutoriel~\cite{ref7} sur la bibliothèque Swing pourra nous servir pour la conception de l'interface graphique. Cette bibliothèque 
		permettra par exemple d'afficher le plateau et d'interagir avec au besoin.
		\\[0.7\baselineskip]

		\item Ce tutoriel~\cite{ref8} sur JUnit montre comment automatiser les tests en Java et comment utiliser les assertions proposées par la librairie. 
		\\[0.7\baselineskip]

	\end{itemize}

