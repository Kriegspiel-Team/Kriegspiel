\documentclass[12pt]{article}
\usepackage[utf8]{inputenc}
\usepackage[french]{babel}
\usepackage{url}
\usepackage[hidelinks,breaklinks]{hyperref}
\usepackage{breakurl}


\input{vc.tex}

\begin{document}

	\title{Projet de Programmation}
	\author{David Cheminade\and Guillaume Desbieys\and Quentin Michaud\and Hubert Mondon}
	\date{}
	\maketitle

	\let\thefootnote\relax
	\footnotetext{Revision~\GITAbrHash, \GITAuthorDate.}

	\section{Etude de l'existant}

		Le jeu de la guerre ou kriegspiel est l'évolution d'une amélioration du jeu d'échecs.
		Le modèle qui nous intéresse est celui de Guy Debord qui l'a breveté en 1965.
		Une application permettant de jouer au jeu a été développée par Radical Software Group (RSG), un collectif de développeurs et d'artistes créé en 2000 dont le but est de travailler sur des programmes expérimentaux.\\[1\baselineskip]
		Cette version du jeu a été développée en Java en utilisant les librairies suivantes

		~~\\

		\begin{itemize}
			\item Java Monkey Engine pour le moteur de jeu
			\item Project Darkstar pour la partie réseau
			\item La modélisation des objets graphiques a été faite via Blender
			~~\\
		\end{itemize}

		L'interface du jeu a été travaillée de sorte à montrer aux joueurs les déplacements possibles pour chaque pièce. De plus les axes de communication sont affichés avec des pointillés pour faciliter la vision du jeu.\\[1\baselineskip]
		L'application propose un mode \"practice\" permettant de jouer seul afin de se familiariser avec les règles du jeu ainsi que l'interface.
		Un second mode permet aux joueurs de s'affronter en ligne afin d'expérimenter de nouvelles stratégies.

		~~\\
		Les développeurs du site RSG ont écarté l'idée d'y implémenter une Intelligence Artificielle puisque le but de leur projet était de pouvoir y apprendre la stratégie face a un adversaire réel.\\[1\baselineskip]
		Malgré les documents décrivant des stratégies de jeu, il n'existe donc pas actuellement d'IA informatique pour ce jeu.

	\section{Elements bibliographiques}
	\begin{itemize}
	
		\item Le livre~\cite{ref1} présente une partie commentée du jeu de la guerre accompagnée de schémas illustrant les positions des unités. De plus on peut trouver les règles officielles proposées par Guy Debord à partir de la page 131.
		~~\\

		\item Le livre Artificial Intelligence for Games~\cite{ref2} aborde dans son 5\up{ème} chapitre la prise de décision d'un point de vue technique. On y trouve le principe de fonctionnement et un exemple d'implémentation de diverses structures de données (arbre de décision, machine à état, arbre de comportement, moteur de règles).
		~~\\

		\item Ce site~\cite{ref3} regroupe l'ensemble des règles du jeu de la guerre et propose une version jouable. Cette version jouable nous a permis de bien comprendre les règles du jeu et de vérifier que notre jeu a bien le même comportement.
		~~\\

		\item Cet ouvrage~\cite{ref4} concernant l'intelligence artificielle propose une grande variété d'idées sur les programmes heuristiques. Il explique comment gérer la prise de décision acceptable mais non optimale, dans les cas où la prise de décision optimale nécessite trop de ressources.
		~~\\

		\item Cette page~\cite{ref5} propose des explications d'algorithmes utilisables dans le cas de jeu se jouant à deux, tour par tour et pour lequel chaque joueur connait la position de son adversaire.
		~~\\
		
		\item Cette page~\cite{ref6} propose un tutoriel très complet sur l'utilisation de Jboss drools, outil qui nous a permis de concevoir notre moteur de règles.
		~~\\
		
		\item Ce tutoriel~\cite{ref7} sur la bibliothèque Swing nous a servi pour la conception de l'interface graphique. Cette bibliothèque permet par exemple d'afficher le plateau et d'interagir avec au besoin.
		~~\\

	\end{itemize}

	\bibliographystyle{unsrt}
	\bibliography{ref.bib}{}
\end{document}