%\section{Etude des besoins}   
\chapter{Etude des besoins}
\markboth{\MakeUppercase{Etude des besoins}}{}

	\section{Besoins fonctionnels}

		\subsection{Réalisation d'un moteur de règles}

			Notre moteur de règles est un système qui doit posséder l'ensemble des règles du jeu afin de pouvoir les appliquer à une situation de jeu.

			\begin{enumerate}

				\item \textbf{Transcrire les règles sous forme logique et les incorporer au moteur.} 
				\\[0.7\baselineskip]
				Priorité : Forte 
				\\[0.7\baselineskip]
				Description : Nous devons transcrire les règles du jeu de façon à ce que le système puisse les comprendre c'est à dire que ces 
				règles doivent être codées à l'aide de conditions logiques. Etant donné une situation de jeu, le moteur de règles doit pouvoir 
				propager les lignes de communications, octroyer les bonus défensifs pour les unités sur les forts, etc.
				\\[0.7\baselineskip]
				Tests : Pour tester notre moteur de règles, nous pouvons utiliser la partie commentée du livre de Guy Debord. Nous pouvons 
				vérifier que les coups qui sont effectués dans cette partie soient bien affichés parmis les coups valides proposés par notre 
				moteur de règles. Cela ne nous garantira pas un fonctionnement parfait du moteur de règles néanmoins ce test pourrait nous 
				aider à trouver des dysfonctionnements. 
				\\[0.7\baselineskip]
				
			\end{enumerate}

		\subsection{Interface graphique}

			\begin{enumerate}

				\item \textbf{Créer une interface de jeu minimale} 
				\\[0.7\baselineskip]
				Priorité : Forte 
				\\[0.7\baselineskip]
				Description : Nous devons créer une interface permettant de représenter de manière très minimaliste le plateau de jeu (position des unités, 
				montagnes, lignes de communication). 
				\\[0.7\baselineskip]
				Tests : On vérifiera l'exactitude du plateau et la cohérence des informations affichées par rapport au modèle interne. 
				\\[0.7\baselineskip]

				\item \textbf{Représentation graphique de la situation} 
				\\[0.7\baselineskip]
				Priorité : Forte 
				\\[0.7\baselineskip]
				Description : Cette interface devra également permettre de visualiser les résultats de l'évaluation de la situation.
				\\[0.7\baselineskip]
				Tests : Bien que cette fonctionnalité sera en elle-même un test de l'évaluateur, son fonctionnement pourra être vérifié par exemple en 
				le faisant cohabiter avec des affichages en console. 
				\\[0.7\baselineskip]

				
			\end{enumerate}

		\subsection{Représentation d'une situation de jeu}

			\begin{enumerate}

				\item \textbf{Créer un système basique de chargement d'une situation de jeu} 
				\\[0.7\baselineskip]
				Priorité : Forte 
				\\[0.7\baselineskip]
				Description : Afin de pouvoir tester les algorithmes du moteur de règles et l'évaluation de positions plus efficacement, nous aurons besoin 
				d'un système permettant de charger des situations de jeu directement depuis un fichier texte, par exemple en ASCII ou autre format simple. 
				Ce fichier devra simplement contenir les coordonnées de tout les pions sur le plateau de jeu. 
				\\[0.7\baselineskip]
				Tests : Il s'agira simplement de comparer les données du fichier de chargement avec celles du plateau de jeu. 
				\\[0.7\baselineskip]
				
			\end{enumerate}

		\subsection{Evaluation statique des positions}

			\begin{enumerate}

				\item \textbf{Déterminer la zone d'influence des unités} 
				\\[0.7\baselineskip]
				Priorité : Moyenne 
				\\[0.7\baselineskip]
				Description : Pour chaque unité, il sera nécessaire de déterminer en fonction des règles incorporées au moteur sa zone d'influence exacte, 
				c'est à dire toutes les cases sur lesquelles l'unité en question pourra se déplacer ou attaquer au tour suivant. Cela permettra de déterminer 
				l'ensemble des coups jouables pour chaque unité. 
				\\[0.7\baselineskip]
				Tests : Placer une cavalerie près d'une ligne de montagnes en présence d'autres unités et vérifier qu'il peut se déplacer normalement sans traverser 
				les différents obstacles. 
				\\[0.7\baselineskip]

				\item \textbf{Calculer les potentiels offensifs et défensifs des unités} 
				\\[0.7\baselineskip]
				Priorité : Moyenne 
				\\[0.7\baselineskip]
				Description : Chaque unité possède d'après les règles ses propres caractéristiques (attaque, défense...) mais le potentiel offensif et défensif 
				d'une unité peut être modifié si d'autres sont en présence directe. Il sera donc nécessaire de calculer ces potentiels pour chaque unité pour en déduire 
				par la suite les zones fortes ou faibles. 
				\\[0.7\baselineskip]
				Tests : Prévoir différentes situations radicalement différentes et noter les potentiels calculés par le jeu disponible sur r-s-g.org. Vérifier ensuite 
				que nous avons les mêmes valeurs pour toutes les situations testées. 
				\\[0.7\baselineskip]

				\item \textbf{Exploiter les données pour déterminer les coefficients de dangeurosité d'une zone} 
				\\[0.7\baselineskip]
				Priorité : Moyenne 
				\\[0.7\baselineskip]
				Description : Après avoir déterminé les zones d'influence et les potentiels offensifs et défensifs des unités, nous pourrons exploiter ces données pour 
				créer un plateau secondaire coefficienté sur lequel seront représentées les zones à éviter ou non. Les coefficients seront calculés pour chaque case en 
				fonction de la proximité des unités et de leur valeur offensive. Dans un premier temps, une seule armée sera prise en compte dans les calculs. Nous pourrons 
				par la suite prendre en compte la seconde armée et adapter les coefficients en fonction de celle-ci. 
				\\[0.7\baselineskip]
				Tests : Prévoir plusieurs situations différentes et déterminer nous-même les zones dangereuses ou interessantes, puis comparer 
				avec le résultat de notre algorithme. 
				\\[0.7\baselineskip]
				
			\end{enumerate}

		\subsection{Evaluation dynamique des positions}

			\begin{enumerate}

				\item \textbf{Déterminer les coefficients de dangeurosité d'une zone sur plusieurs tours} 
				\\[0.7\baselineskip]
				Priorité : Faible 
				\\[0.7\baselineskip]
				Description : De la même manière que l'évaluateur statique déterminera des coefficients pour chaque zone, nous devrons par la suite en reprenant des 
				algorithmes très similaires, redéterminer des coefficients de dangeurosité sur chaque case, en prenant en compte cette fois ci, plusieurs tours de jeu. 
				Le but sera d'obtenir un plateau de jeu coefficienté indiquant les zones avantageuses et désavantageuses pour chaque armée. 
				\\[0.7\baselineskip]
				Tests : Dans un premier temps, en prenant des situations précises, il s'agira de déterminer sur papier (en appliquant les mêmes algorithmes que ceux 
				implémentés dans l'évaluateur) les cases du plateau étant dangereuses pour une armée et avantageuses pour l'autre, et d'en comparer les résultats avec 
				les coefficients obtenus par l'évaluateur. 
				\\[0.7\baselineskip]
				
			\end{enumerate}

		\subsection{Moteur tactique}

			\begin{enumerate}

				\item \textbf{Appliquer une stratégie en fonction des données résultantes de l'évaluation de positions} 
				\\[0.7\baselineskip]
				Priorité : Faible 
				\\[0.7\baselineskip]
				Description : Ce dernier besoin consistera à essayer de mettre en place un système de règles logiques permettant la décision de coups en exploitant les 
				calculs de l'évaluateur dynamique de positions. 
				
			\end{enumerate}
	

	\section{Besoins non fonctionnels}
        
                \subsection{Réaliser une IHM esthétique}
                
                Dans le but de rendre l'utilisation du programme plus agréable visuellement, il serait nécessaire de fournir une interface graphique plus esthétique.
                Il faudra par exemple, rendre le plateau de jeu plus ressemblant à un champ de bataille (définir une image de fond, affiner le quadrillage des cases) attribuer à chaque pièce du plateau une image la représentant, et aussi redéfinir l'affichage des boutons.

                \subsection{Réaliser un éditeur de partie avancé}

                Afin de rendre plus pratique et rapide l'édition d'une situation d'une partie, il serait nécessaire de créer une fonction d'édition directement à partir du plateau de l'IHM, par simple drag and drop.
                De plus, il serait plus pratique d'ajouter la possibilité d'enregistrer les parties éditées.

                \subsection{Affichage des informations plus fluide et pratique}

                Au lieu de fournir des boutons permettant l'affichage des informations calculés par le programme, il serait plus agréable pour l'utilisateur de fournir une fonction affichant ces informations par simple survol des pièces à partir du curseur de la souris.
