%\section{Evolution possible}
\chapter{Evolution possible}
\markboth{\MakeUppercase{Evolution possible}}{}     

	\section{Mise en place d'un joueur minimal}
	
	\paragraph{}
	Actuellement, l'évaluation d'une situation de jeu se fait de façon statique et il n'est pas possible de passer d'une situation de jeu à une autre
	en permettant de jouer un coup. Nous aurions pu concrétiser le projet en mettant en place un joueur très minimal (qui aurait simplement pour but de
	se déplacer vers les arsenaux ennemis par exemple) et ce en mettant en place une boucle de jeu et un moteur décisionnel le plus basique possible.
	Le joueur automatique aurait alors été très sommaire, c'est pourquoi nous nous étions penchés sur de nombreuses possibilités d'évolution.

	\section{Extraction d'informations plus pertinentes}
		
		\paragraph{}
		A ce jour, nous sommes parvenus à extraire une matrice permettant de visualiser les points faibles d'une armée à partir d'une situation de jeu donnée
		dans le cas d'une confrontation directe. (c'est à dire que les deux armées doivent être sur le point de s'attaquer)
		Cette matrice est cependant très insuffisante pour permettre à un joueur automatique de prendre un décision puisque beaucoup d'informations manquent.
		Nous pourrions déduire de cette matrice un vaste ensemble de coups possibles, qu'il faudrait ensuite raffiner en prenant d'autres informations en compte.
	
		\subsection{Raffinement grâce aux lignes de communication}
	
		\paragraph{}
		Pour la suite du projet, si nous nous basions exclusivement sur la matrice permettant la visualisation des points faibles pour prendre une décision,
		le joueur automatique pourrait faire sortir ses unités des lignes de communication, ce qui serait dans la plupart des cas une erreur tactique importante.
		Nous pourrions donc déduire de cette matrice un ensemble de coups potentiellement intéressants et ferions l'intersection avec l'ensemble
		des coups permettant de rester dans les lignes de communication. Nous obtiendrions alors l'ensemble des coups permettant l'attaque des points faibles
		de l'armée adverse tout en restant sur des lignes de comunication alliées.
		Bien que plus efficace que la matrice des points faibles seule, cette solution nécessitera malgré tout de nombreuses autres améliorations avant d'être
		réellement exploitable.
		
		\subsection{Prise en compte des arsenaux}
		
		\paragraph{}
		Si nous nous basions seulement sur la méthode précédente pour créer un joueur automatique, celui-ci serait très vite limité puisque les arsenaux qui sont des
		éléments cruciaux dans la partie ne seraient alors pas pris en compte.
		Il pourrait être intéressant d'ajouter à la liste des coups intéressants les coups permettant de s'approcher d'un arsenal ennemi sans avoir à combattre.
		Pour cela, il faudrait générer k chemins sensiblement différents entre un groupe d'unités et un arsenal, puis analyser pour chacun des k chemins si une 
		zone dangeureuse se trouve entre le groupe d'unité et l'arsenal.
		Il suffit pour cela de récupérer les cases composants les k chemins puis de regarder les valeurs de dangerosité disponibles dans la matrice des zones dangereuses
		que nous avons calculé.
		Si une des valeurs est élevée, alors le chemin doit être abandonné et si toutes les valeurs sont faibles, il peut être intéressant de se diriger vers cet
		arsenal et les cases composant le début du chemin doivent être ajoutées aux déplacements potentiellement intéressants.
		Cette méthode devra également être raffinée grâce à la prise en compte des lignes de communication (de la même façon qu'expliquée précédemment) pour éviter de 
		sortir des lignes en se dirigeant vers un arsenal.

		\subsection{Autres données à prendre en compte}
		
		\paragraph{}
		Après avoir mis en place ces tactiques de base, des tactiques plus compèxes devront être mises en place pour obtenir un joueur minimal.
		Nous ne nous sommes pas penchés en profondeur sur le sujet étant donné qu'il ne s'agissait pas de notre but premier mais avons tout de même quelques idées.
		Il serait intéressant par exemple de mettre en place des stratégies de défense en vérifant qu'il n'existe pas de chemin en ligne droite entre un groupe
		d'unité ennemi et un arsenal allié sur lequel les valeurs défensives seraient faibles.
		Les relais sont également des unités essentielles au niveau statégique, il faudra donc prévoir de les prendre en compte pour les attaquer si possible
		en priorité.
		
		\paragraph{}
		Il faut bien sûr garder à l'esprit que même si toutes ces stratégies étaient correctement implémentées, de grosses failles tactiques persisteraient
		très probablement. Il faudrait par la suite un travail conséquent pour les détecter et mettre en place des algorithme plus efficaces prenant en compte
		tous les paramètres du jeu.
		

	
