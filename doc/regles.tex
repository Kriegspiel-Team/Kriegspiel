%\section{Règles du jeu}  
\chapter{Règles du jeu}
	\markboth{\MakeUppercase{Règles du jeu}}{}

		\section{Apprendre à jouer}
		
		Le jeu de la guerre est un jeu de stratégie au tour par tour se jouant sur un plateau de 500 cases. (25 * 20)
		Le but de celui-ci est de détruire l'armée adverse en la privant de ses arsenaux ou en détruisant toutes ses unités. (Voir les conditions de victoire)
		Vous disposez de plusieurs types d'unités ayant des caractéristiques différentes (attaque, défense, portée, vitesse) que vous devez déplacer 
		en prenant en compte les lignes de communication.
		En effet, si vos unités sortent des lignes de communication, celles-ci ne pourront alors plus bouger et seront donc plus vulnérables.
		Pour accéder aux règles complètes, (caractéristiques des unités, explications plus détaillées) rendez-vous sur le site r-s-g.org~\cite{ref1}
		
		\section{Règles à implémenter}
		Pour toute situation de jeu donnée, voici la liste des règles qui doivent être prises en compte par notre moteur.

		\paragraph{Conditions de victoire}
                 (Il existe trois façons différentes de remporter une partie)
		\begin{enumerate}
		\item Détruire tous les arsenaux adverses.
		\item Détruire toutes les unités de combat adverses.
		\item Détruire les relais et couper toutes les communications des unités adverses.
		\end{enumerate}
		
		\paragraph{Lignes de communications}
		\begin{enumerate}
		\item Les arsenaux propagent des lignes de communication suivant huit axes (nord, nord-est, est, sud-est, sud, sud-ouest, ouest, nord-ouest).
		\item Un arsenal détruit ne propage plus de lignes de communication.
		\item Un relais propage les lignes de communications de la même manière qu'un arsenal, depuis sa position, si il se trouve lui même sur une ligne de communication alliée.
		\item Une unité est connectée si elle se trouve sur une ligne de communication.
		\item Une unité est connectée si elle se trouve sur l'une des huit cases voisines d'une unité alliée connectée.
		\item Une unité déconnectée ne peut ni bouger ni attaquer.
		\item L'attaque et la défense d'une unité déconnectée est égale à 0.(Mais elle peut tout de même bénéficier de la défence apportée par les unités alliées à porté)
		\end{enumerate}
		
		\paragraph{Règles de combat}
		\begin{enumerate}
		\item Chaque type d'unité mobile a une valeur d'attaque/défense/portée qui lui est spécifique. (voir règles précises)
		\item L'attaque subie par une unité se calcule en additionnant toutes les attaques des unités adverses à portée.
		\item Le potentiel défensif d'une unité se calcule en additionnant toutes les défenses des unités alliées à portée.
		\item Même si une unité est à portée d'une autre, sa valeur d'attaque/défense n'est pas ajoutée lors des calculs si un obstacle (montagne) se trouve entre les deux.
		\item Un cavalier ne peut pas charger dans une forteresse.
		\item Une unité peut être détruite si l'attaque subie par celle-ci est supérieur à son potentiel défensif.
		\item Une unité doit battre en retraite si l'attaque subie par celle-ci est égale à son potentiel défensif.
		\end{enumerate}
		
		\paragraph{Bonus défensifs et offensifs}
		\begin{enumerate}
		\item Les cols octroient un bonus défensif de 2 aux infanteries et aux canons.
		\item Les forteresses octroient un bonus défensif de 4 aux infanteries et aux canons.
		\item Si plusieurs cavaliers sont alignés face à une unité adverse et que l'un d'eux est voisin de cette unité, ils sont en situation de charge et bénéficient d'un bonus d'attaque de 3 et peuvent tous attaquer même si ils sont hors de portée.
		\end{enumerate}
		
		\clearpage
