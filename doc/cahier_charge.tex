\documentclass[12pt]{article}
\usepackage[utf8]{inputenc}  
\usepackage[francais]{babel}  
\usepackage{graphicx}

\input{vc.tex}
                 

                            
\begin{document} 
	\title{Cahier des charges}             
	\author{
		David Cheminade
		\and
		Guillaume Desbieys
		\and
		Quentin Michaud
		\and
		Hubert Mondon
	}                       
	\date{}
	\maketitle{}       

	\let\thefootnote\relax
	\footnotetext{Revision~\GITAbrHash, \GITAuthorDate.}                      

	\section{Besoins fonctionnels}    

		\subsection{Réalisation d'un moteur de règles}

			Notre moteur de règles est un système qui doit posséder l'ensemble des règles du jeu afin de pouvoir les appliquer à une situation de jeu.

			\begin{enumerate}

				\item \textbf{Transcrire les règles sous forme logique et les incorporer au moteur.} \\[0.7\baselineskip]
				Priorité : Forte \\[0.7\baselineskip]
				Description : Nous devons transcrire les règles du jeu de façon à ce que le système puisse les comprendre c'est à dire que ces règles doivent être codées à l'aide de conditions logiques. \\[0.7\baselineskip]
				Faisabilité : La transcription des règles ne pose pas de problème au niveau de la faisabilité, il nous faut juste trouver une façon de pouvoir les coder de façon à ce qu'elles puissent être maintenues facilement. La difficulté réside surtout dans le nombre de règles à coder. \\[0.7\baselineskip]
				
				\item \textbf{Le moteur doit pouvoir appliquer les règles à une situation de jeu.} \\[0.7\baselineskip]
				Priorité : Forte \\[0.7\baselineskip]
				Description : Etant donné une situation de jeu, le moteur de règles doit être capable d'appliquer les règles. Il doit pouvoir vérifier que les données du jeu (position des unités) soient en accord avec les règles. Par exemple il ne faudrait pas qu'une unité soit positionnée sur une case de montagne. \\[0.7\baselineskip]

				\item \textbf{Commenter chaque coup en spécifiant la règle l'autorisant ou le refusant.} \\[0.7\baselineskip]
				Priorité : Forte \\[0.7\baselineskip]
				Description : Le moteur de règles doit être capable de pouvoir afficher un message à destination de l'utilisateur afin de lui informer quelles sont les règles qui sont appliquées pour autoriser ou refuser les différents coups. \\[0.7\baselineskip]

				Tests du moteur de règles : Pour tester notre moteur de règles, nous pouvons utiliser la partie commentée du livre de Guy Debord. Nous pouvons vérifier que les coups qui sont effectués dans cette partie soient bien affichés parmis les coups valides proposés par notre moteur de règles. Cela ne nous garantira pas un fonctionnement parfait du moteur de règles néanmoins ce test pourrait nous aider à trouver des disfonctionnements. \\[0.7\baselineskip]
				
			\end{enumerate}

		\subsection{Interface graphique}

			\begin{enumerate}

				\item \textbf{Créer une interface de jeu minimale} \\[0.7\baselineskip]
				Priorité : Forte \\[0.7\baselineskip]
				Description : Nous devons créer une interface permettant de représenter de manière très minimaliste le plateau de jeu (position des unités, montagnes, lignes de communication). \\[0.7\baselineskip]
				Faisabilité : De par sa simplicité, cette interface devrait être facile a réaliser. \\[0.7\baselineskip]
				Tests : On vérifiera l'exactitude du plateau et la cohérence des informations affichées par rapport au modèle interne. \\[0.7\baselineskip]

				\item \textbf{Représentation graphique de la situation} \\[0.7\baselineskip]
				Priorité : Forte \\[0.7\baselineskip]
				Description : Cette interface devra également permettre de visualiser l'analyse effectuée par l'évaluateur de situation. \\[0.7\baselineskip]
				Faisabilité : Il s'agira simplement de lire des matrices et d'affecter par exemple des couleurs aux différentes valeurs des cases. \\[0.7\baselineskip]
				Tests : Bien que cette fonctionnalité sera en elle-même un test de l'évaluateur, son fonctionnement poura être vérifié par exemple en le faisant cohabiter avec des affichages en console. \\[0.7\baselineskip]

				
			\end{enumerate}

		\subsection{Représentation d'une situation de jeu}

			\begin{enumerate}

				\item \textbf{Créer un système basique de chargement d'une situation de jeu} \\[0.7\baselineskip]
				Priorité : Forte \\[0.7\baselineskip]
				Description : Afin de pouvoir tester les algorithmes du moteur de règles et de l'évaluateur de position plus efficacement, nous aurons besoin d'un système permettant de charger des situations de jeu directement depuis un fichier texte, par exemple en ASCII ou autre format simple. Ce fichier devra simplement contenir les coordonnées de tout les pions sur le plateau de jeu. \\[0.7\baselineskip]
				Faisabilité : Le fichier devra pouvoir être lu correctement par notre programme, et être retranscrit correctement dans les données du plateau de jeu. \\[0.7\baselineskip]
				Tests : Il s'agira simplement de comparer les données du fichier de chargement avec celles du plateau de jeu. \\[0.7\baselineskip]
				
			\end{enumerate}

		\subsection{Evaluation statique des positions}

			\begin{enumerate}

				\item \textbf{Déterminer la zone d'influence des unités} \\[0.7\baselineskip]
				Priorité : Moyenne \\[0.7\baselineskip]
				Description : Pour chaque unité, il sera nécessaire de déterminer en fonction des règles incorporées au moteur sa zone d'influence exacte, c'est à dire toutes les cases sur lesquelles l'unité en question pourra se déplacer ou attaquer au tour suivant. Cela permettra de déterminer l'ensemble des coups jouables pour chaque unité. \\[0.7\baselineskip]
				Faisabilité : La difficulté de ce besoin réside principalement dans la détection d'obstacles. En effet, les unités ne peuvent par exemple pas passer de l'autre côté de montagnes ou de lignes d'unités même si le nombre de cases qui sépare l'unité de la case de destination est inférieure à la capacité de l'unité en question. \\[0.7\baselineskip]
				Tests : Placer une cavalerie près d'une ligne de montagnes en présence d'autres unités et vérifier qu'il peut se déplacer normalement sans traverser les différents obstacles. \\[0.7\baselineskip]

				\item \textbf{Calculer les potentiels offensifs et défensifs des unités} \\[0.7\baselineskip]
				Priorité : Moyenne \\[0.7\baselineskip]
				Description : Chaque unité possède d'après les règles ses propres caractéristiques (attaque, défense...) mais le potentiel offensif et défensif d'une unité peut être modifié si d'autres sont en présence directe. Il sera donc nécessaire de calculer ces potentiels pour chaque unité pour en déduire par la suite les zones fortes ou faibles. \\[0.7\baselineskip]
				Faisabilité : Il existe un grand nombre de règles régissant le calcul de ces différents potentiels, il faudra donc faire de nombreux tests pour vérifier que ces calculs ne sont pas érronés car cela pourrait avoir de plus un impact très négatif sur le reste du projet. \\[0.7\baselineskip]
				Tests : Prévoir différentes situations radicalement différentes et noter les potentiels calculés par le jeu disponible sur r-s-g.org. Vérifier ensuite que nous avons les mêmes valeurs pour toutes les situations testées. \\[0.7\baselineskip]

				\item \textbf{Exploiter les données pour déterminer les coefficients de prédominance d'une zone} \\[0.7\baselineskip]
				Priorité : Moyenne \\[0.7\baselineskip]
				Description : Après avoir déterminé les zones d'influence et les potentiels offensifs et défensifs des unités, nous pourrons exploiter ces données pour créer un plateau secondaire coefficienté sur lequel seront représentées les zones à éviter ou non. Les coefficients seront calculés pour chaque case en fonction de la proximité des unités et de leur valeur offensive. Dans un premier temps, une seule armée sera prise en compte dans les calculs. Nous pourrons par la suite prendre en compte la seconde armée et adapter les coefficients en fonction de celle-ci. \\[0.7\baselineskip]
				Faisabilité : Il sera difficile de trouver un algorithme efficace permettant de générer des coefficients réellement utiles. Il faudra probablement prendre en compte de nombreux paramètres différents. \\[0.7\baselineskip]
				Tests : Prévoir plusieurs situations différentes et déterminer nous-même les zones dangereuses ou interessantes, puis comparer avec le résultat de notre algorithme. \\[0.7\baselineskip]
				
			\end{enumerate}

		\subsection{Evaluation dynamique des positions}

			\begin{enumerate}

				\item \textbf{Déterminer les coefficients de prédominance d'une zone sur plusieurs tours} \\[0.7\baselineskip]
				Priorité : Faible \\[0.7\baselineskip]
				Description : De la même manière que l'évaluateur statique déterminera des coefficients pour chaque zone, nous devrons par la suite en reprenant des algorithmes très similaires, redéterminer des coefficients de prédominance sur chaque case, en prenant en compte cette fois ci, plusieurs tours de jeu. Le but sera d'obtenir un plateau de jeu coefficienté indiquant les zones avantageuses et désavantageuses pour chaque armée. \\[0.7\baselineskip]
				Faisabilité : Ce besoin étant très similaire au dernier de l'évaluateur statique (tout en étant cette fois dynamique) le point délicat sera donc le même, c'est à dire la création d'algorithmes de calculs efficaces qui affecteront aux cases du plateau des coefficients en fonctions de nombreux paramètres, tout en rajoutant la prise en compte de plusieurs tours de jeu. \\[0.7\baselineskip]
				Tests : Dans un premier temps, en prenant des situations précises, il s'agira de déterminer sur papier (en appliquant les mêmes algorithmes que ceux implémentés dans l'évaluateur) les cases du plateau étant dangereuses pour une armée et avantageuses pour l'autre, et d'en comparer les résultats avec les coefficients obtenus par l'évaluateur. \\[0.7\baselineskip]
				
			\end{enumerate}

		\subsection{Moteur tactique}

			\begin{enumerate}

				\item \textbf{Appliquer une stratégie en fonction des données résultantes de l'évaluation de positions} \\[0.7\baselineskip]
				Priorité : Faible \\[0.7\baselineskip]
				Description : Ce dernier besoin consistera à essayer de mettre en place un système de règles logiques permettant la décision de coups en exploitant les calculs de l'évaluateur dynamique de positions. \\[0.7\baselineskip]
				Faisabilité : Le premier point délicat sera la mise en place de ce système de règles de décisons, qui devra interpréter logiquement les données (plateau coefficienté) extraites des calculs de l'évaluateur dynamique. Un second point délicat sera de réussir à calculer l'ordre de priorité des coups (au nombre de cinq par tour) afin que le déplacement des pions soit réalisable (par exemple, un groupe de cinq soldats dont un seul est placé sur la ligne de communication ne pourra être déplacé que par une séquence de coup définie dans un ordre précis). \\[0.7\baselineskip]
				Tests : De nombreux tests seront à réaliser si nous arrivons à ce dernier besoin. Il s'agira de vérifier tous nos algortihmes sur le plus grands nombre de situations possibles et différentes, afin de trouver un maximum d'erreurs à corriger. Nous devrons réaliser tous les types de tests possibles sur notre programme et essayer de provoquer des erreurs en faisant des tests avec des valeurs inapropriées. \\[0.7\baselineskip]
				
			\end{enumerate}

	\section{Besoins non fonctionnels} 

		\begin{itemize}
                        \item Réaliser une IHM esthetique.(pièces représentées par des images, plateau ressemblant à un champ de bataille, redefinition de l'image des boutons)
                        \item Realiser un editeur de partie à partir de l'IHM.
                        \item Permettre l'affichage d'information des unités par simple survole du curseur sur l'unité
			\item Créer un historique des parties jouées
		\end{itemize}

\end{document}                                 
