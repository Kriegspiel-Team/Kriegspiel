\chapter{Bilan}
\markboth{\MakeUppercase{Bilan}}{}	

	\section{Difficultés rencontrées}   

		\subsection{A. Utilisation de Drools}
		Il est vrai qu'en début de projet, nous ne savions pas encore très bien utiliser Drools. Nous avions testé sur de petits exemple mais cela a été plus compliqué
		à mettre en place sur un projet de cette ampleur. Nous avions en effet beaucoup de code java dans les règles et cela manquait donc de clarté. Nous avons
		donc clarifié notre code au fur et à mesure du projet et avons appris petit à petit à nous servir de Drools correctement. 
		
		\subsection{B. Gestion des attaques}
		Les combats étant définis par de nombreuses règles, il a été assez complexe de les implémenter sans erreur.
		En effet, le potentiel d'attaque reçu par une unité se calcule en fonction des unités adverses à portée, en vérifiant bien sûr que celle-ci est connectée aux lignes
		de communication et qu'aucun obstacle se situe entre les deux.
		Il faut également prendre en compte les bonus défensifs des cols et des forteresses, mais aussi la possibilité de charge pour les cavaliers.
		Rappelons que pour bénéficier de la charge, il faut aligner un ou plusieurs cavaliers face à une unité adverse.
		Cela leur octroie alors un bonus d'attaque et permet aux cavaliers en bout de ligne d'attaquer même s'ils ne sont pas à portée sauf dans le cas ou l'un des cavaliers
		se situe sur une forteresse.
		Il fut donc assez compliqué d'implémenter toutes ces règles sans faire d'erreur notamment la charge qui est une règle spéciale pour laquelle il y a
		plusieurs exceptions.
		
	\section{Rendu final}
	
		Nous sommes parvenus à réaliser tous nos besoins fonctionnels et à extraire une matrice des points faibles plutôt intéressante. 
		Nous sommes donc satisfait sur ce point mais aurions aimé avoir eu le temps de concrétiser réellement le projet en mettant en place un joueur minimal
		en implémentant une boucle de jeu. Nous avons cependant préféré peaufiner notre code et tester plus en profondeur afin d'avoir un rendu plus stable.
		
	\clearpage

	\section{Conclusion}
	
		Il est vrai que nous n'imaginions pas au moment du choix du sujet que le travail à réaliser en amont du joueur artificiel serait si conséquent.
		Nous sommes cependant parvenus à extraire des informations qui pourront être utilisées dans une future prise de décision et cet aspect fut
		intéressant à réaliser. Ces données extraites des situations de jeu pourront par ailleurs être utilisées pour la création d'un joueur automatique 
		si le sujet est proposé l'année prochaine.
		
	\clearpage
