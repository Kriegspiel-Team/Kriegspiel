%\section{Introduction}

\chapter{Introduction}
\markboth{\MakeUppercase{Introduction}}{}


	\paragraph{}
	Le jeu de la guerre a été inventé par Guy Debord en 1965 dans le but de faire apparaître les différentes stratégies de la guerre sur un jeu de plateau. 
	Les règles y sont donc nombreuses et le jeu est assez complexe ce qui permet la mise en application de stratégies très variées.
	En raison de ces nombreuses règles et de sa complexité notoire, il n'existe à ce jour aucun joueur contrôlé par ordinateur permettant de jouer à ce jeu.
	
	\paragraph{}
	Réaliser un joueur automatique pour ce jeu est une tâche conséquente nécessitant un travail important en amont pour préparer la prise de décision.
	Dans le cadre de ce PDP, c'est ce travail de préparation à la prise de décision que nous avons effectué, plutôt que la prise de décision elle même.
	Notre projet consiste ainsi à pouvoir évaluer de façon statique une situation de jeu en prenant en compte toutes les règles qui s'appliquent via le moteur de règles.
	En d'autres termes, nous avons dû extraire des informations utiles d'une situation spécifique en prenant en compte les règles du jeu dans le but futur de pouvoir prendre une décision acceptable même si non optimisée.
	
	\paragraph{}
	Nous expliquerons dans ce rapport quelle a été notre approche pour venir à bout de ce projet et détaillerons par ailleurs notre architecture et nos
	choix d'implémentation. Nous présenterons également le rendu obtenu en détaillant le travail qui a été effectué et les options manquantes, puis
	nous nous pencherons sur les tests qui ont été faits pour vérifier le bon fonctionnement de notre projet.
