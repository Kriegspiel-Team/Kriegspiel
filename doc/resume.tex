\documentclass[12pt]{article}
\usepackage[utf8]{inputenc}  
\usepackage[francais]{babel}                      
\title{Cahier des charges}                  
\author{
        David Cheminade
        \and
        Guillaume Desbieys
        \and
        Quentin Michaud
        \and
        Hubert Mondon
}                       
\date{}                             
\begin{document} 
        \maketitle{}                             

				Suite à la réunion du 17 janvier 2014, il nous a été demandé de réaliser une synthèse de ce qui a été dit. Afin d'être le plus clair et concis possible, nous avons décidé de présenter les principales idées évoquées sous forme de liste, plutôt que de les présenter sous une forme déstructurée.
				
				\section{Points importants abordés en réunion :}

                \begin{enumerate}
                
                        \item Lire le livre de Debord. (Analyse des règles et de la partie commentée)
                        \item Lire "AI for games". (Exemples d'IA et informations générales)
                        \item Lire dans les grandes lignes le livre de Clausewitz. (Stratégies de guerre)
                        \item Jouer au jeu et bien connaître les stratégies de base.
                        \item Regarder le site R-S-G.org en détail.
                        \item Choisir des mots clefs plus spécifiques pour nos recherches. (exemple : "Domination game")
                        \item Créer un fichier de notes partagé ou un wiki.
                        \item Faire un choix entre reprendre le code source de R-S-G.org / Créer une surcouche / Créer une interface minimale.
                        \item Etape 1 : Créer le moteur de règles avec commentaires indiquant quelle règle permet de jouer le coup.
                        \item Etape 2 : Analyse du champ d'action des pièces.
                        \item Etape 3 : Analyse de la situation (offensive/défensive) et création d'une ou plusieurs matrices coefficientées.
                        \item Etape 4 : Intégrer des stratégies avancées. (mouvements de plusieurs groupes d'unités par exemple)
                        \item Ne pas intégrer de méthode d'apprentissage.
                        \item Ne pas utiliser la méthode du parcours de graphe.
                       
                \end{enumerate}   

\end{document}    
